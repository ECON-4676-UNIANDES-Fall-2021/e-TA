% Options for packages loaded elsewhere
\PassOptionsToPackage{unicode}{hyperref}
\PassOptionsToPackage{hyphens}{url}
%
\documentclass[
]{article}
\usepackage{lmodern}
\usepackage{amssymb,amsmath}
\usepackage{ifxetex,ifluatex}
\ifnum 0\ifxetex 1\fi\ifluatex 1\fi=0 % if pdftex
  \usepackage[T1]{fontenc}
  \usepackage[utf8]{inputenc}
  \usepackage{textcomp} % provide euro and other symbols
\else % if luatex or xetex
  \usepackage{unicode-math}
  \defaultfontfeatures{Scale=MatchLowercase}
  \defaultfontfeatures[\rmfamily]{Ligatures=TeX,Scale=1}
\fi
% Use upquote if available, for straight quotes in verbatim environments
\IfFileExists{upquote.sty}{\usepackage{upquote}}{}
\IfFileExists{microtype.sty}{% use microtype if available
  \usepackage[]{microtype}
  \UseMicrotypeSet[protrusion]{basicmath} % disable protrusion for tt fonts
}{}
\makeatletter
\@ifundefined{KOMAClassName}{% if non-KOMA class
  \IfFileExists{parskip.sty}{%
    \usepackage{parskip}
  }{% else
    \setlength{\parindent}{0pt}
    \setlength{\parskip}{6pt plus 2pt minus 1pt}}
}{% if KOMA class
  \KOMAoptions{parskip=half}}
\makeatother
\usepackage{xcolor}
\IfFileExists{xurl.sty}{\usepackage{xurl}}{} % add URL line breaks if available
\IfFileExists{bookmark.sty}{\usepackage{bookmark}}{\usepackage{hyperref}}
\hypersetup{
  pdftitle={e-TA 1: Version control with Git(Hub)},
  hidelinks,
  pdfcreator={LaTeX via pandoc}}
\urlstyle{same} % disable monospaced font for URLs
\usepackage[margin=1in]{geometry}
\usepackage{color}
\usepackage{fancyvrb}
\newcommand{\VerbBar}{|}
\newcommand{\VERB}{\Verb[commandchars=\\\{\}]}
\DefineVerbatimEnvironment{Highlighting}{Verbatim}{commandchars=\\\{\}}
% Add ',fontsize=\small' for more characters per line
\usepackage{framed}
\definecolor{shadecolor}{RGB}{248,248,248}
\newenvironment{Shaded}{\begin{snugshade}}{\end{snugshade}}
\newcommand{\AlertTok}[1]{\textcolor[rgb]{0.94,0.16,0.16}{#1}}
\newcommand{\AnnotationTok}[1]{\textcolor[rgb]{0.56,0.35,0.01}{\textbf{\textit{#1}}}}
\newcommand{\AttributeTok}[1]{\textcolor[rgb]{0.77,0.63,0.00}{#1}}
\newcommand{\BaseNTok}[1]{\textcolor[rgb]{0.00,0.00,0.81}{#1}}
\newcommand{\BuiltInTok}[1]{#1}
\newcommand{\CharTok}[1]{\textcolor[rgb]{0.31,0.60,0.02}{#1}}
\newcommand{\CommentTok}[1]{\textcolor[rgb]{0.56,0.35,0.01}{\textit{#1}}}
\newcommand{\CommentVarTok}[1]{\textcolor[rgb]{0.56,0.35,0.01}{\textbf{\textit{#1}}}}
\newcommand{\ConstantTok}[1]{\textcolor[rgb]{0.00,0.00,0.00}{#1}}
\newcommand{\ControlFlowTok}[1]{\textcolor[rgb]{0.13,0.29,0.53}{\textbf{#1}}}
\newcommand{\DataTypeTok}[1]{\textcolor[rgb]{0.13,0.29,0.53}{#1}}
\newcommand{\DecValTok}[1]{\textcolor[rgb]{0.00,0.00,0.81}{#1}}
\newcommand{\DocumentationTok}[1]{\textcolor[rgb]{0.56,0.35,0.01}{\textbf{\textit{#1}}}}
\newcommand{\ErrorTok}[1]{\textcolor[rgb]{0.64,0.00,0.00}{\textbf{#1}}}
\newcommand{\ExtensionTok}[1]{#1}
\newcommand{\FloatTok}[1]{\textcolor[rgb]{0.00,0.00,0.81}{#1}}
\newcommand{\FunctionTok}[1]{\textcolor[rgb]{0.00,0.00,0.00}{#1}}
\newcommand{\ImportTok}[1]{#1}
\newcommand{\InformationTok}[1]{\textcolor[rgb]{0.56,0.35,0.01}{\textbf{\textit{#1}}}}
\newcommand{\KeywordTok}[1]{\textcolor[rgb]{0.13,0.29,0.53}{\textbf{#1}}}
\newcommand{\NormalTok}[1]{#1}
\newcommand{\OperatorTok}[1]{\textcolor[rgb]{0.81,0.36,0.00}{\textbf{#1}}}
\newcommand{\OtherTok}[1]{\textcolor[rgb]{0.56,0.35,0.01}{#1}}
\newcommand{\PreprocessorTok}[1]{\textcolor[rgb]{0.56,0.35,0.01}{\textit{#1}}}
\newcommand{\RegionMarkerTok}[1]{#1}
\newcommand{\SpecialCharTok}[1]{\textcolor[rgb]{0.00,0.00,0.00}{#1}}
\newcommand{\SpecialStringTok}[1]{\textcolor[rgb]{0.31,0.60,0.02}{#1}}
\newcommand{\StringTok}[1]{\textcolor[rgb]{0.31,0.60,0.02}{#1}}
\newcommand{\VariableTok}[1]{\textcolor[rgb]{0.00,0.00,0.00}{#1}}
\newcommand{\VerbatimStringTok}[1]{\textcolor[rgb]{0.31,0.60,0.02}{#1}}
\newcommand{\WarningTok}[1]{\textcolor[rgb]{0.56,0.35,0.01}{\textbf{\textit{#1}}}}
\usepackage{graphicx}
\makeatletter
\def\maxwidth{\ifdim\Gin@nat@width>\linewidth\linewidth\else\Gin@nat@width\fi}
\def\maxheight{\ifdim\Gin@nat@height>\textheight\textheight\else\Gin@nat@height\fi}
\makeatother
% Scale images if necessary, so that they will not overflow the page
% margins by default, and it is still possible to overwrite the defaults
% using explicit options in \includegraphics[width, height, ...]{}
\setkeys{Gin}{width=\maxwidth,height=\maxheight,keepaspectratio}
% Set default figure placement to htbp
\makeatletter
\def\fps@figure{htbp}
\makeatother
\setlength{\emergencystretch}{3em} % prevent overfull lines
\providecommand{\tightlist}{%
  \setlength{\itemsep}{0pt}\setlength{\parskip}{0pt}}
\setcounter{secnumdepth}{-\maxdimen} % remove section numbering
\ifluatex
  \usepackage{selnolig}  % disable illegal ligatures
\fi

\title{e-TA 1: Version control with Git(Hub)}
\author{true}
\date{}

\begin{document}
\maketitle

Hello! Welcome to e-TAs, your on-line help for ECON 4676. So you want to
learn how to use Github. In this e-TA, you will learn the intricacies
and specifics of working with Github and strategies to keep your file
version control under your control. The present issue focuses on the
basic operations of \texttt{Git} and \texttt{Github}. The core material
was extracted from tutorials at the
\href{https://www.uiuc-bdeep.org/about}{BDEEP group} at
\href{http://www.ncsa.illinois.edu/site}{NCSA} and
\href{https://grantmcdermott.com/}{Prof.~Grant McDermott} \footnote{If
  you have comments, suggestions, etc. please submit a pull request ;).}

\hypertarget{motivation}{%
\section{Motivation}\label{motivation}}

\hypertarget{github-solves-this-problem}{%
\subsection{Git(Hub) solves this
problem}\label{github-solves-this-problem}}

\hypertarget{git}{%
\subsubsection{Git}\label{git}}

\begin{itemize}
\tightlist
\item
  Git is a distributed version control system. (Wait, what?)
\item
  Okay, try this: Imagine if Dropbox and the ``Track changes'' feature
  in MS Word had a baby. Git would be that baby.
\item
  In fact, it's even better than that because Git is optimized for the
  things that economists and data scientists spend a lot of time working
  on (e.g.~code).
\item
  There is a learning curve, but I promise you it's worth it.
\end{itemize}

\hypertarget{github}{%
\subsubsection{GitHub}\label{github}}

\begin{itemize}
\tightlist
\item
  It's important to realize that Git and GitHub are distinct things.
\item
  GitHub is an online hosting platform that provides an array of
  services built on top of the Git system. (Similar platforms include
  Bitbucket and GitLab.)
\item
  Just like we don't \emph{need} Rstudio to run R code, we don't
  \emph{need} GitHub to use Git\ldots{} But it will make our lives so
  much easier.
\end{itemize}

\hypertarget{repositories}{%
\section{Repositories}\label{repositories}}

When you begin a new coding project, you will want to create a new
repository. This repository will grow to hold all of the important files
for your project.

\hypertarget{create-a-repository}{%
\subsubsection{Create a repository}\label{create-a-repository}}

You can create a new repository by clicking the green ``new repository''
button on your github homepage.

\begin{itemize}
\item
  Step 1: Name your repository Give your repository an accurate and
  concise name. You want the project to be recognizable! You can also
  add an optional description to summarize the goal of your project.
\item
  Step 2: Public or private Most repositories should be made public
  unless you have a specific reason to make it private. A public
  repository can be viewed by anyone but you can choose who can commit
  to it. Since it can be seen by others, it is possible to collaborate
  with others if you need help. A private repository can only be viewed
  by you and you are the only person who can commit to it, but you are
  able to manually add other collaborators as well. Private
  reposoitories are only available for paid accounts.
\item
  Step 3: Add a README Finally, select ``initialize with a README''. You
  should only keep this unchecked if you are creating a repository from
  an existing repository.
\item
  Step 4: Licence The license dropdown allows you to select a license if
  you are sharing open source software.
\item
  Step 5: gitignore dropdown allows you to choose any file types that
  will not be committed
\end{itemize}

\hypertarget{repositories-1}{%
\section{Repositories}\label{repositories-1}}

Congratulations! Now you have created your own repository! Once you have
a repository on GitHub, here is how you link it to your computer. This
section is more code based side, with the specific commands you will
need to use to get version control working for you!

\begin{enumerate}
\def\labelenumi{\arabic{enumi})}
\item
  First you need to get your repo url
\item
  Open up your command line tool, and write \texttt{git\ clone} with the
  url you copied. For example:
\end{enumerate}

\begin{Shaded}
\begin{Highlighting}[]
\NormalTok{$ }\FunctionTok{git}\NormalTok{ clone https://github.com/ignaciomsarmiento/test\_repo.git}
\end{Highlighting}
\end{Shaded}

Now the repo is in your local computer. You can see for example your
commit history (hit spacebar to scroll down or q to exit).

\begin{Shaded}
\begin{Highlighting}[]
\NormalTok{$ }\FunctionTok{git}\NormalTok{ log}
\end{Highlighting}
\end{Shaded}

or what has changed?

\begin{Shaded}
\begin{Highlighting}[]
\NormalTok{$ }\FunctionTok{git}\NormalTok{ status}
\end{Highlighting}
\end{Shaded}

to add files

Stage (``add'') a file or group of files.

\begin{Shaded}
\begin{Highlighting}[]
\NormalTok{$ }\FunctionTok{git}\NormalTok{ add NAME{-}OF{-}FILE{-}OR{-}FOLDER}
\end{Highlighting}
\end{Shaded}

You can use
\href{https://ryanstutorials.net/linuxtutorial/wildcards.php}{wildcard}
characters to stage a group of files (e.g.~sharing a common prefix).
There are a bunch of useful flag options too:

\begin{itemize}
\tightlist
\item
  Stage all files.
\end{itemize}

\begin{Shaded}
\begin{Highlighting}[]
\NormalTok{$ }\FunctionTok{git}\NormalTok{ add {-}A}
\end{Highlighting}
\end{Shaded}

\begin{itemize}
\tightlist
\item
  Stage updated files only (modified or deleted, but not new).
\end{itemize}

\begin{Shaded}
\begin{Highlighting}[]
\NormalTok{$ }\FunctionTok{git}\NormalTok{ add {-}u}
\end{Highlighting}
\end{Shaded}

\begin{itemize}
\tightlist
\item
  Stage new files only (not updated).
\end{itemize}

\begin{Shaded}
\begin{Highlighting}[]
\NormalTok{$ }\FunctionTok{git}\NormalTok{ add .}
\end{Highlighting}
\end{Shaded}

Commit your changes.

\begin{Shaded}
\begin{Highlighting}[]
\NormalTok{$ }\FunctionTok{git}\NormalTok{ commit {-}m }\StringTok{"Helpful message"}
\end{Highlighting}
\end{Shaded}

Pull from the upstream repository (i.e.~GitHub).

\begin{Shaded}
\begin{Highlighting}[]
\NormalTok{$ }\FunctionTok{git}\NormalTok{ pull}
\end{Highlighting}
\end{Shaded}

Push any local changes that you've committed to the upstream repo
(i.e.~GitHub).

\begin{Shaded}
\begin{Highlighting}[]
\NormalTok{$ }\FunctionTok{git}\NormalTok{ push origin master}
\end{Highlighting}
\end{Shaded}

\texttt{origin} is a shorthand name for the remote repository that a
project was originally cloned from. and \texttt{master} the branch we
are pushing to.

\hypertarget{forking}{%
\section{Forking}\label{forking}}

Forking is going to be the way to contribute to my lectures, e-TAs, etc,
and get those precious participation points

\begin{itemize}
\item
  In fact, if you fork a repo then you are really creating a copy of it.
\item
  Forking a repo on GitHub is
  \href{https://help.github.com/articles/fork-a-repo/}{very simple};
  just click the ``Fork'' button in the top-right corner of said repo.
\item
  This will create an independent copy of the repo under your GitHub
  account.
\item
  Once you fork a repo, you are free to do anything you want to it.
  (It's yours.) However, forking --- in combination with pull requests
  --- is actually how much of the world's software is developed. For
  example:
\item
  Outside user \emph{B} forks \emph{A}'s repo. She adds a new feature
  (or fixes a bug she's identified) and then
  \href{https://help.github.com/articles/creating-a-pull-request-from-a-fork/}{issues
  an upstream pull request}.
\item
  \emph{A} is notified and can then decide whether to merge \emph{B}'s
  contribution with the main project.
\end{itemize}

\hypertarget{branches}{%
\section{Branches}\label{branches}}

\textbf{Incomplete Section:} the best pull requests for this section get
some bonus points =)

\hypertarget{if-things-go-wrong}{%
\section{If Things go Wrong}\label{if-things-go-wrong}}

\textbf{Q: What happens when something goes wrong?}

\textbf{A: Think: ``Oh shit, Git!''} - Seriously:
\url{http://ohshitgit.com/}.

\textbf{Q: What happens when something goes horribly wrong?}

\textbf{A: Burn it down and start again.} -
\url{http://happygitwithr.com/burn.html} - This is a great advantage of
Git's distributed nature. If something goes horribly wrong, there's
usually an intact version somewhere else.

\end{document}
